%% Generated by Sphinx.
\def\sphinxdocclass{report}
\documentclass[letterpaper,10pt,italian]{sphinxmanual}
\ifdefined\pdfpxdimen
   \let\sphinxpxdimen\pdfpxdimen\else\newdimen\sphinxpxdimen
\fi \sphinxpxdimen=.75bp\relax
\ifdefined\pdfimageresolution
    \pdfimageresolution= \numexpr \dimexpr1in\relax/\sphinxpxdimen\relax
\fi
%% let collapsible pdf bookmarks panel have high depth per default
\PassOptionsToPackage{bookmarksdepth=5}{hyperref}

\PassOptionsToPackage{booktabs}{sphinx}
\PassOptionsToPackage{colorrows}{sphinx}

\PassOptionsToPackage{warn}{textcomp}
\usepackage[utf8]{inputenc}
\ifdefined\DeclareUnicodeCharacter
% support both utf8 and utf8x syntaxes
  \ifdefined\DeclareUnicodeCharacterAsOptional
    \def\sphinxDUC#1{\DeclareUnicodeCharacter{"#1}}
  \else
    \let\sphinxDUC\DeclareUnicodeCharacter
  \fi
  \sphinxDUC{00A0}{\nobreakspace}
  \sphinxDUC{2500}{\sphinxunichar{2500}}
  \sphinxDUC{2502}{\sphinxunichar{2502}}
  \sphinxDUC{2514}{\sphinxunichar{2514}}
  \sphinxDUC{251C}{\sphinxunichar{251C}}
  \sphinxDUC{2572}{\textbackslash}
\fi
\usepackage{cmap}
\usepackage[T1]{fontenc}
\usepackage{amsmath,amssymb,amstext}
\usepackage{babel}



\usepackage{tgtermes}
\usepackage{tgheros}
\renewcommand{\ttdefault}{txtt}



\usepackage[Sonny]{fncychap}
\ChNameVar{\Large\normalfont\sffamily}
\ChTitleVar{\Large\normalfont\sffamily}
\usepackage{sphinx}

\fvset{fontsize=auto}
\usepackage{geometry}


% Include hyperref last.
\usepackage{hyperref}
% Fix anchor placement for figures with captions.
\usepackage{hypcap}% it must be loaded after hyperref.
% Set up styles of URL: it should be placed after hyperref.
\urlstyle{same}


\usepackage{sphinxmessages}
\setcounter{tocdepth}{1}



\title{L\textquotesingle{}Esame}
\date{15 dic 2023}
\release{0.1}
\author{Emanuele Donno}
\newcommand{\sphinxlogo}{\vbox{}}
\renewcommand{\releasename}{Release}
\makeindex
\begin{document}

\ifdefined\shorthandoff
  \ifnum\catcode`\=\string=\active\shorthandoff{=}\fi
  \ifnum\catcode`\"=\active\shorthandoff{"}\fi
\fi

\pagestyle{empty}
\sphinxmaketitle
\pagestyle{plain}
\sphinxtableofcontents
\pagestyle{normal}
\phantomsection\label{\detokenize{index::doc}}


\sphinxstepscope


\chapter{Codice Deontologico}
\label{\detokenize{capitoli/codice/codice_deontologico:codice-deontologico}}\label{\detokenize{capitoli/codice/codice_deontologico:id1}}\label{\detokenize{capitoli/codice/codice_deontologico::doc}}
\sphinxAtStartPar
Questa è la versione digitale (\sphinxstylestrong{non ufficiale}) del \sphinxstylestrong{Codice Deontologico} scaricabile in formato PDF dalla \sphinxhref{https://www.cni.it/cni/codice-deontologico}{pagina ufficiale} del \sphinxstylestrong{Consiglio Nazionale degli Ingegneri Italiani (CNI)}.

\sphinxAtStartPar
Come riportato nel \sphinxhref{https://www.cni.it/images/News/2023/Codice\_deontologico\_CNI\_14-06-2023\_DEF.rev.1.pdf}{documento ufficiale}, il Codice Deontologico degli Ingegneri Italiani è stato:

\sphinxAtStartPar
\sphinxtitleref{Approvato in data 1° dicembre 2006, adeguato in data 9 aprile 2014, integrato in data 23 marzo 2022 e da ultimo aggiornato in data 14 giugno 2023.}

\sphinxAtStartPar
Esso è composto da 23 articoli, divisi in 7 Capi:


\begin{savenotes}\sphinxattablestart
\sphinxthistablewithglobalstyle
\centering
\begin{tabulary}{\linewidth}[t]{TTT}
\sphinxtoprule
\sphinxtableatstartofbodyhook
\sphinxAtStartPar
\sphinxstylestrong{CAPO}
&
\sphinxAtStartPar
\sphinxstylestrong{Titolo}
&
\sphinxAtStartPar
\sphinxstylestrong{Articoli}
\\
\sphinxhline
\sphinxAtStartPar
I
&
\sphinxAtStartPar
Parte Generale
&
\sphinxAtStartPar
1 \sphinxhyphen{} 2
\\
\sphinxhline
\sphinxAtStartPar
II
&
\sphinxAtStartPar
Doveri Generali
&
\sphinxAtStartPar
3 \sphinxhyphen{} 12
\\
\sphinxhline
\sphinxAtStartPar
III
&
\sphinxAtStartPar
Rapporti Interni
&
\sphinxAtStartPar
13 \sphinxhyphen{} 16
\\
\sphinxhline
\sphinxAtStartPar
IV
&
\sphinxAtStartPar
Rapporti Esterni
&
\sphinxAtStartPar
17 \sphinxhyphen{} 19
\\
\sphinxhline
\sphinxAtStartPar
V
&
\sphinxAtStartPar
Rapporti con l’Ordine
&
\sphinxAtStartPar
20
\\
\sphinxhline
\sphinxAtStartPar
VI
&
\sphinxAtStartPar
Incompatibilità
&
\sphinxAtStartPar
21 \sphinxhyphen{} 22
\\
\sphinxhline
\sphinxAtStartPar
VII
&
\sphinxAtStartPar
Disposizioni Finali
&
\sphinxAtStartPar
23
\\
\sphinxbottomrule
\end{tabulary}
\sphinxtableafterendhook\par
\sphinxattableend\end{savenotes}


\bigskip\hrule\bigskip



\section{\sphinxstyleemphasis{Capo I \sphinxhyphen{} Parte Generale}}
\label{\detokenize{capitoli/codice/codice_deontologico:capo-i-parte-generale}}

\subsection{Articolo 01 \sphinxhyphen{} Principi Generali}
\label{\detokenize{capitoli/codice/codice_deontologico:articolo-01-principi-generali}}\begin{description}
\sphinxlineitem{\sphinxstylestrong{1.1}}
\sphinxAtStartPar
La professione di Ingegnere deve essere esercitata nel rispetto delle leggi e regolamenti emanati dallo Stato e/o dai suoi organi, dei principi costituzionali e dell’ordinamento comunitario.

\sphinxlineitem{\sphinxstylestrong{1.2}}
\sphinxAtStartPar
Le prestazioni professionali dell’Ingegnere devono essere svolte tenendo conto della tutela della vita e della salute dell’uomo.

\end{description}


\subsection{Articolo 02 \sphinxhyphen{} Finalità e ambito di applicazione}
\label{\detokenize{capitoli/codice/codice_deontologico:articolo-02-finalita-e-ambito-di-applicazione}}\begin{description}
\sphinxlineitem{\sphinxstylestrong{2.1}}
\sphinxAtStartPar
Le presenti norme si applicano agli iscritti ad ogni settore e in ogni sezione dell’albo, in qualunque forma gli stessi svolgano l’attività di Ingegnere e sono finalizzate alla tutela dei valori e interessi generali connessi all’esercizio professionale e del decoro della professione.

\sphinxlineitem{\sphinxstylestrong{2.2}}
\sphinxAtStartPar
Le previsioni del presente Codice deontologico devono intendersi riferite a tutti i titoli professionali stabiliti dall’\sphinxstyleemphasis{art.45} del Decreto del Presidente della Repubblica \sphinxstyleemphasis{5 giugno 2001 n.328}, e sinteticamente indicati con il termine “\sphinxstylestrong{Ingegnere}”.

\sphinxlineitem{\sphinxstylestrong{2.3}}
\sphinxAtStartPar
Chiunque eserciti la professione di Ingegnere in Italia è impegnato a rispettare e far rispettare il presente Codice Deontologico, anche se cittadino di altro Stato ed anche nel caso di prestazioni transfrontaliere occasionali temporanee.

\sphinxlineitem{\sphinxstylestrong{2.4}}
\sphinxAtStartPar
Il rispetto delle presenti norme è dovuto anche per prestazioni rese all’estero, unitamente al rispetto delle norme etico\sphinxhyphen{}deontologiche vigenti nel paese in cui viene svoltala prestazione professionale.

\end{description}


\bigskip\hrule\bigskip



\section{\sphinxstyleemphasis{Capo II \sphinxhyphen{} Doveri Generali}}
\label{\detokenize{capitoli/codice/codice_deontologico:capo-ii-doveri-generali}}

\subsection{Articolo 03 \sphinxhyphen{} Doveri dell’Ingegnere}
\label{\detokenize{capitoli/codice/codice_deontologico:articolo-03-doveri-dellingegnere}}\begin{description}
\sphinxlineitem{\sphinxstylestrong{3.1}}
\sphinxAtStartPar
L’Ingegnere sostiene e difende il decoro e la reputazione della propria professione.

\sphinxlineitem{\sphinxstylestrong{3.2}}
\sphinxAtStartPar
L’Ingegnere accetta le responsabilità connesse ai propri compiti e dà garanzia di poter rispondere degli atti professionali svolti.

\sphinxlineitem{\sphinxstylestrong{3.3}}
\sphinxAtStartPar
L’Ingegnere deve adempiere agli impegni assunti con diligenza, perizia e prudenza e deve informare la propria attività professionale ai principi di integrità, lealtà, chiarezza, correttezza e qualità della prestazione.

\sphinxlineitem{\sphinxstylestrong{3.4}}
\sphinxAtStartPar
L’Ingegnere ha il dovere di conservare la propria autonomia tecnica e intellettuale, rispetto a qualsiasi forma di pressione e condizionamento esterno di qualunque natura.

\sphinxlineitem{\sphinxstylestrong{3.5}}
\sphinxAtStartPar
Costituisce infrazione disciplinare l’evasione fiscale e/o previdenziale definitivamente accertata.

\end{description}


\subsection{Articolo 04 \sphinxhyphen{} Correttezza}
\label{\detokenize{capitoli/codice/codice_deontologico:articolo-04-correttezza}}\begin{description}
\sphinxlineitem{\sphinxstylestrong{4.1}}
\sphinxAtStartPar
L’Ingegnere rifiuta di accettare incarichi e di svolgere attività professionali nei casi in cui ritenga di non avere adeguata preparazione e competenza e/o quelli per i quali ritenga di non avere adeguati mezzi ed organizzazione per l’adempimento degli impegni assunti.

\sphinxlineitem{\sphinxstylestrong{4.2}}
\sphinxAtStartPar
L’Ingegnere sottoscrive solo le prestazioni professionali che abbia svolto e/o diretto; non sottoscrive le prestazioni professionali in forma paritaria unitamente a persone che per norme vigenti non le possono svolgere.

\sphinxlineitem{\sphinxstylestrong{4.3}}
\sphinxAtStartPar
Costituisce altresì illecito disciplinare il comportamento dell’Ingegnere che agevoli, o, in qualsiasi altro modo diretto o indiretto, renda possibile a soggetti non abilitati o sospesi l’esercizio abusivo dell’attività di Ingegnere o consenta che tali soggetti ne possano ricavare benefici economici, anche se limitatamente al periodo di eventuale sospensione dall’esercizio.

\sphinxlineitem{\sphinxstylestrong{4.4}}
\sphinxAtStartPar
Qualsiasi dichiarazione, attestazione o asseverazione resa dall’Ingegnere deve essere preceduta da verifiche, al fine di renderle coerenti con la realtà dei fatti e dei luoghi.

\sphinxlineitem{\sphinxstylestrong{4.5}}
\sphinxAtStartPar
L’Ingegnere non può accettare da terzi compensi diretti o indiretti, oltre a quelli dovutigli dal committente, senza comunicare a questi natura, motivo ed entità ed aver avuto per iscritto autorizzazione alla riscossione.

\sphinxlineitem{\sphinxstylestrong{4.6}}
\sphinxAtStartPar
L’Ingegnere non cede ad indebite pressioni e non accetta di rendere la prestazione in caso di offerte o proposte di remunerazioni, compensi o utilità di qualsiasi genere che possano pregiudicare la sua indipendenza di giudizio.

\sphinxlineitem{\sphinxstylestrong{4.7}}
\sphinxAtStartPar
L’Ingegnere verifica preliminarmente la correttezza e la legittimità dell’attività professionale e rifiuta di formulare offerte, accettare incarichi o di prestare la propria attività quando possa fondatamente desumere da elementi conosciuti che la sua attività concorra a operazioni illecite o illegittime e palesemente incompatibili coi principi di liceità, moralità, efficienza e qualità.

\end{description}


\subsection{Articolo 05 \sphinxhyphen{} Legalità}
\label{\detokenize{capitoli/codice/codice_deontologico:articolo-05-legalita}}\begin{description}
\sphinxlineitem{\sphinxstylestrong{5.1}}
\sphinxAtStartPar
Costituisce illecito disciplinare lo svolgimento di attività professionale in mancanza di titolo in settori o sezioni diversi da quelli di competenza o in periodo di sospensione.

\sphinxlineitem{\sphinxstylestrong{5.2}}
\sphinxAtStartPar
Il comportamento dell’Ingegnere che certifica, dichiara o attesta la falsa esistenza di requisiti e/o presupposti per la legittimità dei conseguenti atti e provvedimenti amministrativi costituisce violazione disciplinare.

\sphinxlineitem{\sphinxstylestrong{5.3}}
\sphinxAtStartPar
Costituisce grave violazione deontologica, lesiva della categoria professionale, ogni forma di partecipazione o contiguità in affari illeciti a qualunque titolo collegati o riconducibili alla criminalità organizzata o comunque a soggetti dediti al malaffare.

\end{description}


\subsection{Articolo 06 \sphinxhyphen{} Riservatezza}
\label{\detokenize{capitoli/codice/codice_deontologico:articolo-06-riservatezza}}\begin{description}
\sphinxlineitem{\sphinxstylestrong{6.1}}
\sphinxAtStartPar
L’Ingegnere deve mantenere il segreto professionale sulle informazioni assunte nell’esecuzione dell’incarico professionale.

\sphinxlineitem{\sphinxstylestrong{6.2}}
\sphinxAtStartPar
L’Ingegnere è tenuto a garantire le condizioni per il rispetto del dovere di riservatezza a coloro che hanno collaborato alla prestazione professionale.

\end{description}


\subsection{Articolo 07 \sphinxhyphen{} Formazione e aggiornamento}
\label{\detokenize{capitoli/codice/codice_deontologico:articolo-07-formazione-e-aggiornamento}}\begin{description}
\sphinxlineitem{\sphinxstylestrong{7.1}}
\sphinxAtStartPar
L’Ingegnere deve costantemente migliorare le proprie conoscenze per mantenere le proprie capacità professionali ad un livello adeguato allo sviluppo della tecnologia, della legislazione, e dello stato dell’arte della cultura professionale.

\sphinxlineitem{\sphinxstylestrong{7.2}}
\sphinxAtStartPar
L’Ingegnere deve costantemente aggiornare le proprie competenze professionali seguendo i percorsi di formazione professionale continua così come previsto dalla legge.

\end{description}


\subsection{Articolo 08 \sphinxhyphen{} Assicurazione professionale}
\label{\detokenize{capitoli/codice/codice_deontologico:articolo-08-assicurazione-professionale}}\begin{description}
\sphinxlineitem{\sphinxstylestrong{8.1}}
\sphinxAtStartPar
Nei casi previsti dalla legge l’Ingegnere, a tutela del committente, è tenuto a stipulare idonea assicurazione per i rischi derivanti dall’esercizio dell’attività professionale.

\sphinxlineitem{\sphinxstylestrong{8.2}}
\sphinxAtStartPar
L’Ingegnere, al momento dell’assunzione dell’incarico, è tenuto a rendere noti al committente gli estremi della polizza stipulata per la responsabilità professionale ed il relativo massimale.

\end{description}


\subsection{Articolo 09 \sphinxhyphen{} Pubblicità informativa}
\label{\detokenize{capitoli/codice/codice_deontologico:articolo-09-pubblicita-informativa}}\begin{description}
\sphinxlineitem{\sphinxstylestrong{9.1}}
\sphinxAtStartPar
La pubblicità deve rispettare la dignità ed il decoro della professione, deve essere finalizzata alla informazione relativamente ai servizi offerti dal professionista e può riguardare l’attività professionale, le specializzazioni ed i titoli posseduti, la struttura dello studio ed i compensi richiesti per le varie prestazioni.

\sphinxlineitem{\sphinxstylestrong{9.2}}
\sphinxAtStartPar
Le informazioni devono essere trasparenti, veritiere, corrette e non devono essere equivoche, ingannevoli o denigratorie.

\end{description}


\subsection{Articolo 10 \sphinxhyphen{} Rapporti con il committente}
\label{\detokenize{capitoli/codice/codice_deontologico:articolo-10-rapporti-con-il-committente}}\begin{description}
\sphinxlineitem{\sphinxstylestrong{10.1}}
\sphinxAtStartPar
L’Ingegnere deve sempre operare nel legittimo interesse del committente, e informare la propria attività ai principi di integrità, lealtà, riservatezza nonché fedeltà al mandato ricevuto.

\end{description}


\subsection{Articolo 11 \sphinxhyphen{} Incarichi e compensi}
\label{\detokenize{capitoli/codice/codice_deontologico:articolo-11-incarichi-e-compensi}}\begin{description}
\sphinxlineitem{\sphinxstylestrong{11.1}}
\sphinxAtStartPar
L’Ingegnere al momento dell’affidamento dell’incarico deve definire con chiarezza i termini dell’incarico conferito e deve pattuire il compenso con il committente, rendendo noto il grado di complessità della prestazione e fornendo tutte le informazioni utili circa gli oneri ipotizzabili correlati o correlabili all’incarico stesso.

\sphinxlineitem{\sphinxstylestrong{11.2}}
\sphinxAtStartPar
L’Ingegnere è tenuto a comunicare al committente eventuali situazioni o circostanze che possano modificare il compenso inizialmente pattuito, indicando l’entità della variazione.

\sphinxlineitem{\sphinxstylestrong{11.3}}
\sphinxAtStartPar
La misura del compenso è correlata all’importanza dell’opera e al decoro della professione ai sensi dell’art. 2233 del codice civile e deve essere resa nota al committente, comprese spese, oneri e contributi. Il compenso relativo alle prestazioni professionali di cui alla legge 21 aprile 2023 n.49 deve essere proporzionato alla quantità e alla qualità del lavoro svolto, al contenuto e alla caratteristiche della prestazione professionale, nonchè conforme ai compensi fissati dai decreti ministeriali, ai sensi dell’art.1 della legge citata.

\sphinxlineitem{\sphinxstylestrong{11.4}}
\sphinxAtStartPar
I compensi professionali previsti nei modelli standard di convenzione, concordati tra imprese e Consiglio Nazionale degli Ingegneri si presumono equi fino a prova contraria, ai sensi dell’art.6 della \sphinxstyleemphasis{legge 21 aprile 2023 n.49}.

\sphinxlineitem{\sphinxstylestrong{11.5}}
\sphinxAtStartPar
L’Ingegnere può fornire prestazioni professionali a titolo gratuito solo in casi particolari quando sussistano valide motivazioni ideali ed umanitarie.

\sphinxlineitem{\sphinxstylestrong{11.6}}
\sphinxAtStartPar
Possono considerarsi prestazioni professionali non soggette a remunerazione tutti quegli interventi di aiuto rivolti a colleghi Ingegneri che, o per limitate esperienze dovute alla loro giovane età o per situazioni professionali gravose, si vengono a trovare in difficoltà.

\end{description}


\subsection{Articolo 12 \sphinxhyphen{} Svolgimento delle prestazioni}
\label{\detokenize{capitoli/codice/codice_deontologico:articolo-12-svolgimento-delle-prestazioni}}\begin{description}
\sphinxlineitem{\sphinxstylestrong{12.1}}
\sphinxAtStartPar
L’incarico professionale deve essere svolto compiutamente, con espletamento di tutte le prestazioni pattuite, tenendo conto degli interessi del committente.

\sphinxlineitem{\sphinxstylestrong{12.2}}
\sphinxAtStartPar
L’Ingegnere deve informare il committente di ogni potenziale confitto di interesse che potrebbe sorgere durante lo svolgimento della prestazione.

\sphinxlineitem{\sphinxstylestrong{12.3}}
\sphinxAtStartPar
L’Ingegnere deve avvertire tempestivamente il committente in caso di interruzione o di rinuncia all’incarico, in modo da non provocare pregiudizio allo stesso.

\sphinxlineitem{\sphinxstylestrong{12.4}}
\sphinxAtStartPar
L’Ingegnere è inoltre tenuto ad informare il committente, nel caso abbia rapporti di interesse su materiali o procedimenti costruttivi proposti per lavori attinenti al suo incarico professionale, quando la natura e la presenza di tali rapporti possano ingenerare sospetto di parzialità professionale o violazione di norme di etica.

\sphinxlineitem{\sphinxstylestrong{12.5}}
\sphinxAtStartPar
L’Ingegnere è tenuto a consegnare al committente i documenti dallo stesso ricevuti o necessari all’espletamento dell’incarico nei termini pattuiti, quando quest’ultimo ne faccia richiesta.

\end{description}


\bigskip\hrule\bigskip



\section{\sphinxstyleemphasis{Capo III \sphinxhyphen{} Rapporti Interni}}
\label{\detokenize{capitoli/codice/codice_deontologico:capo-iii-rapporti-interni}}

\subsection{Articolo 13}
\label{\detokenize{capitoli/codice/codice_deontologico:articolo-13}}\begin{description}
\sphinxlineitem{\sphinxstylestrong{13.1}}
\sphinxAtStartPar
L’Ingegnere deve improntare i rapporti professionali con i colleghi alla massima lealtà e correttezza, allo scopo di affermare una comune cultura ed identità professionale pur neidifferenti settori in cui si articola la professione.

\sphinxlineitem{\sphinxstylestrong{13.2}}
\sphinxAtStartPar
Costituisce infrazione alla regola deontologica l’utilizzo di espressioni sconvenienti od offensive negli scritti e nell’attività professionale in genere, sia nei confronti dei colleghi che nei confronti delle controparti e dei terzi.

\sphinxlineitem{\sphinxstylestrong{13.3}}
\sphinxAtStartPar
L’Ingegnere deve astenersi dal porre in essere azioni che possano ledere, con critiche denigratorie o in qualsiasi altro modo, la reputazione di colleghi o di altri professionisti.

\sphinxlineitem{\sphinxstylestrong{13.4}}
\sphinxAtStartPar
L’Ingegnere non deve mettere in atto comportamenti scorretti finalizzati a sostituire in un incarico un altro Ingegnere o altro tecnico, già incaricato per una specifica prestazione.

\sphinxlineitem{\sphinxstylestrong{13.5}}
\sphinxAtStartPar
L’Ingegnere che sia chiamato a subentrare in un incarico già affidato ad altri potrà accettarlo solo dopo che il committente abbia comunicato ai primi incaricati la revoca dell’incarico per iscritto; dovrà inoltre informare per iscritto i professionisti a cui subentra e il Consiglio dell’Ordine.

\sphinxlineitem{\sphinxstylestrong{13.6}}
\sphinxAtStartPar
In caso di subentro ad altri professionisti in un incarico l’Ingegnere subentrante deve fare in modo di non arrecare danni alla committenza ed al collega a cui subentra.

\sphinxlineitem{\sphinxstylestrong{13.7}}
\sphinxAtStartPar
L’Ingegnere sostituito deve adoperarsi affinché la successione del mandato avvenga senza danni per il committente, fornendo al nuovo professionista tutti gli elementi per permettergli la prosecuzione dell’incarico.

\sphinxlineitem{\sphinxstylestrong{13.8}}
\sphinxAtStartPar
L’Ingegnere sottoscrive prestazioni professionali con incarico affidato congiuntamente a più professionisti, in forma collegiale o in gruppo, solo quando siano rispettati e specificati i limiti di competenza professionale, i campi di attività e i limiti di responsabilità dei singoli membri del collegio o del gruppo. Tali limiti dovranno essere dichiarati sin dall’inizio della collaborazione.

\sphinxlineitem{\sphinxstylestrong{13.9}}
\sphinxAtStartPar
L’Ingegnere collabora con i colleghi e li supporta, ove richiesto, nel caso subiscano pressioni lesive della loro dignità personale e della categoria.

\end{description}


\subsection{Articolo 14 \sphinxhyphen{} Rapporti con collaboratori}
\label{\detokenize{capitoli/codice/codice_deontologico:articolo-14-rapporti-con-collaboratori}}\begin{description}
\sphinxlineitem{\sphinxstylestrong{14.1}}
\sphinxAtStartPar
L’Ingegnere può ricorrere sotto la propria direzione e responsabilità a collaboratori e, più in generale, all’utilizzazione di una organizzazione stabile.

\sphinxlineitem{\sphinxstylestrong{14.2}}
\sphinxAtStartPar
I rapporti fra Ingegneri e collaboratori sono improntati alla massima correttezza.

\sphinxlineitem{\sphinxstylestrong{14.3}}
\sphinxAtStartPar
L’Ingegnere assume la piena responsabilità della organizzazione della struttura che utilizza per eseguire l’incarico affidatogli, nonché del prodotto della organizzazione stessa; l’Ingegnere si assume la responsabilità dei collaboratori per i quali deve definire, seguire e controllare il lavoro svolto e da svolgere.

\sphinxlineitem{\sphinxstylestrong{14.4}}
\sphinxAtStartPar
L’Ingegnere nell’espletare l’incarico assunto si impegna ad evitare ogni forma di collaborazione che possa identificarsi con un subappalto non autorizzato del lavoro intellettuale o che porti allo sfruttamento di esso; deve inoltre rifiutarsi di legittimare il lavoro abusivo.

\sphinxlineitem{\sphinxstylestrong{14.5}}
\sphinxAtStartPar
L’Ingegnere deve improntare il rapporto con collaboratori e tirocinanti alla massima chiarezza e trasparenza.

\sphinxlineitem{\sphinxstylestrong{14.6}}
\sphinxAtStartPar
Nei rapporti con i collaboratori e i dipendenti, l’Ingegnere è tenuto ad assicurare ad essi condizioni di lavoro e compensi adeguati.

\sphinxlineitem{\sphinxstylestrong{14.7}}
\sphinxAtStartPar
Nei rapporti con i tirocinanti, l’Ingegnere è tenuto a prestare il proprio insegnamento professionale e a compiere quanto necessario per assicurare ad essi il sostanziale adempimento della pratica professionale, sia dal punto di vista tecnico/scientifico, sia dal punto di vista delle regole deontologiche.

\sphinxlineitem{\sphinxstylestrong{14.8}}
\sphinxAtStartPar
Parimenti l’Ingegnere tirocinante deve improntare il rapporto con il professionista, presso il quale svolge il tirocinio, alla massima correttezza, astenendosi dal porre in essere qualsiasi atto o condotta diretti ad acquisire in proprio i clienti dello studio presso il quale ha svolto il tirocinio.

\end{description}


\subsection{Articolo 15 \sphinxhyphen{} Concorrenza}
\label{\detokenize{capitoli/codice/codice_deontologico:articolo-15-concorrenza}}\begin{description}
\sphinxlineitem{\sphinxstylestrong{15.1}}
\sphinxAtStartPar
La concorrenza è libera e deve svolgersi nel rispetto delle norme deontologiche secondo i principi fissati dalla normativa e dall’ordinamento comunitario.

\sphinxlineitem{\sphinxstylestrong{15.2}}
\sphinxAtStartPar
L’Ingegnere si deve astenere dal ricorrere a mezzi incompatibili con la propria dignità per ottenere incarichi professionali, come l’esaltazione delle proprie qualità a denigrazione delle altrui o fornendo vantaggi o assicurazioni esterne al rapporto professionale.

\sphinxlineitem{\sphinxstylestrong{15.3}}
\sphinxAtStartPar
È sanzionabile disciplinarmente la pattuizione di compensi manifestamente inadeguati alla prestazione da svolgere. In caso di accettazione di incarichi con corrispettivo che si presuma anormalmente basso, l’Ingegnere potrà essere chiamato a dimostrare il rispetto dei principi di efficienza e qualità della prestazione. La violazione, da parte del professionista, dell’obbligo di convenire o di preventivare un compenso che sia giusto, equo e proporzionato alla prestazione professionale richiesta e determinato in applicazione dei parametri previsti dai pertinenti decreti ministeriali, è sanzionata a giudizio del Consiglio di disciplina territoriale, ai sensi dell’art.5, comma 5, della \sphinxstyleemphasis{legge 21 aprile 2023 n.49}.

\sphinxlineitem{\sphinxstylestrong{15.4}}
\sphinxAtStartPar
La violazione dell’obbligo di avvertire il cliente, nei soli rapporti in cui la convenzione, il contratto o comunque qualsiasi accordo con il cliente siano predisposti esclusivamente dal professionista, che il compenso per la prestazione professionale deve rispettare in ogni caso, pena la nullità della pattuizione, i criteri stabiliti dalle disposizioni della legge 21 aprile 2023 n. 49 e dalle altre leggi in vigore è sanzionata a giudizio del Consiglio di disciplina territoriale, ai sensi dell’art.5, comma 5, della legge citata.

\sphinxlineitem{\sphinxstylestrong{15.5}}
\sphinxAtStartPar
L’illecita concorrenza può manifestarsi in diverse forme:
\begin{quote}
\begin{enumerate}
\sphinxsetlistlabels{\alph}{enumi}{enumii}{}{.}%
\item {} 
\sphinxAtStartPar
critiche denigratorie sul comportamento professionale di un collega;

\end{enumerate}

\sphinxAtStartPar
b. attribuzione a sé della paternità di un lavoro eseguito in
collaborazione senza che sia chiarito l’effettivo apporto dei
collaboratori;

\sphinxAtStartPar
c. attribuzione a se stessi del risultato della prestazione professionale
di altro professionista;

\sphinxAtStartPar
d. utilizzazione della propria posizione o delle proprie conoscenze
presso Amministrazioni od Enti Pubblici per acquisire incarichi
professionali direttamente o per interposta persona;

\sphinxAtStartPar
e. partecipazione come consulente presso enti banditori o come membro di commissioni giudicatrici di concorsi che non abbiano avuto esito conclusivo per accettare incarichi inerenti alla
progettazione che è stata oggetto del concorso;

\sphinxAtStartPar
f. abuso di mezzi pubblicitari della propria attività professionale e che
possano ledere in vario modo la dignità della professione.
\end{quote}

\end{description}


\subsection{Articolo 16 \sphinxhyphen{} Attività in forma associativa o societaria}
\label{\detokenize{capitoli/codice/codice_deontologico:articolo-16-attivita-in-forma-associativa-o-societaria}}\begin{description}
\sphinxlineitem{\sphinxstylestrong{16.1}}
\sphinxAtStartPar
Nel caso in cui l’attività professionale, anche di tipo interdisciplinare, sia svolta in forma associativa o societaria nei modi e nei termini di cui alle norme vigenti, le prestazioni professionali devono essere rese sotto la direzione e responsabilità di uno o più soci/associati, il cui nome deve essere preventivamente comunicato al committente.

\sphinxlineitem{\sphinxstylestrong{16.2}}
\sphinxAtStartPar
Gli Ingegneri che intendono esercitare l’attività in forma associata, esclusiva o non esclusiva, devono stabilire per iscritto i termini dei reciproci impegni e la durata del rapporto professionale e, nel caso di incarichi congiunti, devono rispettare i campi e i limiti di responsabilità dei singoli membri del collegio o del gruppo ed a dichiarare tali limiti sin dall’inizio della collaborazione

\sphinxlineitem{\sphinxstylestrong{16.3}}
\sphinxAtStartPar
Nel caso di associazione professionale è disciplinarmente responsabile soltanto l’Ingegnere o gli Ingegneri a cui si riferiscano i fatti specifici commessi.

\sphinxlineitem{\sphinxstylestrong{16.4}}
\sphinxAtStartPar
La forma dell’esercizio professionale non muta le responsabilità professionali derivanti dall’operato dell’Ingegnere nei confronti della committenza e della collettività. Del comportamento dell’Ingegnere nell’ambito dell’attività della società di cui è socio, risponde deontologicamente anche la società iscritta all’Albo.

\end{description}


\bigskip\hrule\bigskip



\section{\sphinxstyleemphasis{Capo IV \sphinxhyphen{} Rapporti Esterni}}
\label{\detokenize{capitoli/codice/codice_deontologico:capo-iv-rapporti-esterni}}

\subsection{Articolo 17 \sphinxhyphen{} Rapporti con le istituzioni}
\label{\detokenize{capitoli/codice/codice_deontologico:articolo-17-rapporti-con-le-istituzioni}}\begin{description}
\sphinxlineitem{\sphinxstylestrong{17.1}}
\sphinxAtStartPar
L’Ingegnere deve astenersi dall’avvalersi, in qualunque forma, per lo svolgimento degli incarichi professionali della collaborazione retribuita dei dipendenti delle Istituzioni se non espressamente a tal fine autorizzati.

\sphinxlineitem{\sphinxstylestrong{17.2}}
\sphinxAtStartPar
L’Ingegnere che sia in rapporti di parentela, familiarità o frequentazione con coloro che rivestono incarichi od operano nelle Istituzioni deve astenersi dal vantare tale rapporto al fine di trarre utilità di qualsiasi natura nella sua attività professionale.

\sphinxlineitem{\sphinxstylestrong{17.3}}
\sphinxAtStartPar
L’Ingegnere che assume cariche istituzionali, o sia nominato in una commissione o giuria, deve svolgere il proprio mandato evitando qualsiasi abuso, diretto o per interposta persona, dei poteri inerenti la carica ricoperta per trarre comunque vantaggi per sé o per altri; non deve, inoltre, vantare tale incarico al fine di trarne utilità nella propria attività professionale. Nello stesso modo, ove sia in rapporti di qualsiasi natura con componenti di commissioni aggiudicatici, non deve vantare tali rapporti per trarre vantaggi di qualsiasi natura per sé o per altri derivanti da tale circostanza.

\end{description}


\subsection{Articolo 18 \sphinxhyphen{} Rapporti con la collettività}
\label{\detokenize{capitoli/codice/codice_deontologico:articolo-18-rapporti-con-la-collettivita}}\begin{description}
\sphinxlineitem{\sphinxstylestrong{18.1}}
\sphinxAtStartPar
L’Ingegnere è personalmente responsabile della propria opera nei confronti della committenza e la sua attività professionale deve essere svolta tenendo conto preminentemente della tutela della collettività.

\end{description}


\subsection{Articolo 19 \sphinxhyphen{} Rapporti con il territorio}
\label{\detokenize{capitoli/codice/codice_deontologico:articolo-19-rapporti-con-il-territorio}}\begin{description}
\sphinxlineitem{\sphinxstylestrong{19.1}}
\sphinxAtStartPar
L’Ingegnere nell’esercizio della propria attività cerca soluzioni ai problemi a lui posti, che siano compatibili con il principio dello sviluppo sostenibile, mirando alla massima valorizzazione delle risorse naturali, al minimo consumo del territorio e al minimo spreco delle fonti energetiche.

\sphinxlineitem{\sphinxstylestrong{19.2}}
\sphinxAtStartPar
Nella propria attività l’Ingegnere è tenuto, nei limiti delle sue funzioni, ad evitare che vengano arrecate all’ambiente nel quale opera alterazioni che possano influire negativamente sull’equilibrio ecologico e sulla conservazione dei beni culturali, artistici, storici e del paesaggio.

\sphinxlineitem{\sphinxstylestrong{19.3}}
\sphinxAtStartPar
L’Ingegnere non può progettare o dirigere opere abusive o difformi alle norme e regolamenti vigenti.

\end{description}


\bigskip\hrule\bigskip



\section{\sphinxstyleemphasis{Capo V \sphinxhyphen{} Rapporti con l’Ordine}}
\label{\detokenize{capitoli/codice/codice_deontologico:capo-v-rapporti-con-l-ordine}}

\subsection{Articolo 20 \sphinxhyphen{} Rapporti con l’Ordine e con gli organismi di autogoverno}
\label{\detokenize{capitoli/codice/codice_deontologico:articolo-20-rapporti-con-lordine-e-con-gli-organismi-di-autogoverno}}\begin{description}
\sphinxlineitem{\sphinxstylestrong{20.1}}
\sphinxAtStartPar
L’appartenenza dell’Ingegnere all’Ordine professionale comporta il dovere di collaborare con il Consiglio dell’Ordine. Ogni Ingegnere ha pertanto l’obbligo, se convocato dal Consiglio dell’Ordine o dal suo Presidente, di presentarsi e di fornire tutti i chiarimenti richiesti.

\sphinxlineitem{\sphinxstylestrong{20.2}}
\sphinxAtStartPar
L’Ingegnere deve provvedere regolarmente e tempestivamente agli adempimentieconomici dovuti nei confronti dell’Ordine.

\sphinxlineitem{\sphinxstylestrong{20.3}}
\sphinxAtStartPar
L’Ingegnere si adegua alle deliberazioni del Consiglio dell’Ordine, se assunte nell’esercizio delle relative competenze istituzionali.

\sphinxlineitem{\sphinxstylestrong{20.4}}
\sphinxAtStartPar
L’Ingegnere che abbia ricevuto una nomina a seguito di una segnalazione da parte dell’Ordine, della Consulta/Federazione o del CNI, deve:
\begin{enumerate}
\sphinxsetlistlabels{\alph}{enumi}{enumii}{}{.}%
\item {} 
\sphinxAtStartPar
comunicare tempestivamente al Consiglio le nomine ricevute in rappresentanza su segnalazione dello stesso o di altri organismi;

\item {} 
\sphinxAtStartPar
svolgere il mandato limitatamente alla durata prevista di esso;

\item {} 
\sphinxAtStartPar
accettare la riconferma consecutiva dello stesso incarico solo nei casi ammessi dal Consiglio o da altro organismo nominante;

\item {} 
\sphinxAtStartPar
prestare la propria opera in forma continuativa per l’intera durata del mandato, seguendo assiduamente e diligentemente i lavori che il suo svolgimento comporta, segnalando al Consiglio dell’Ordine con sollecitudine le violazioni di norme deontologiche delle quali sia venuto a conoscenza nell’adempimento dell’incarico comunque ricevuto;

\item {} 
\sphinxAtStartPar
presentare tempestivamente le proprie dimissioni nel caso di impossibilità a mantenere l’impegno assunto;

\item {} 
\sphinxAtStartPar
controllare la perfetta osservanza delle norme che regolano i lavori a cui si partecipa.

\end{enumerate}

\sphinxlineitem{\sphinxstylestrong{20.5}}
\sphinxAtStartPar
L’Ingegnere è tenuto a rispettare le disposizioni di legge e regolamentari in materia elettorale, ivi incluse quelle delegate al Consiglio Nazionale degli Ingegneri. La violazione delle suddette disposizioni, laddove finalizzata ad anteporre interessi privati a quelli della categoria professionale e a compromettere, per l’effetto, la corretta composizione, il tempestivo insediamento o il regolare funzionamento degli organi di autogoverno della professione, configura un illecito disciplinare. Costituisce, in particolare, grave illecito disciplinare l’inosservanza, da parte dell’Ingegnere che intenda candidarsi a ricoprire la carica di Consigliere territoriale dell’Ordine o di Consigliere nazionale, del limite di mandati elettorali consecutivi stabilito all’art. 2 del Decreto del Presidente della Repubblica 8 luglio 2005 n.169 e dalla normativa vigente.

\end{description}


\bigskip\hrule\bigskip



\section{\sphinxstyleemphasis{Capo VI \sphinxhyphen{} Incompatibilità}}
\label{\detokenize{capitoli/codice/codice_deontologico:capo-vi-incompatibilita}}

\subsection{Articolo 21 \sphinxhyphen{} Incompatibilità}
\label{\detokenize{capitoli/codice/codice_deontologico:articolo-21-incompatibilita}}\begin{description}
\sphinxlineitem{\sphinxstylestrong{21.1}}
\sphinxAtStartPar
L’Ingegnere non svolge prestazioni professionali in condizioni di incompatibilità con il proprio stato giuridico, né quando il proprio interesse o quello del committente o datore di lavoro siano in contrasto con i suoi doveri professionali.

\sphinxlineitem{\sphinxstylestrong{21.2}}
\sphinxAtStartPar
Si manifesta incompatibilità anche nel contrasto con i propri doveri professionali nel caso di partecipazioni a concorsi le cui condizioni del bando siano state giudicate dal Consiglio Nazionale degli Ingegneri o dagli Ordini (per i soli concorsi provinciali), pregiudizievoli ai diritti o al decoro dell’Ingegnere, sempre che sia stata emessa formale diffida e che questa sia stata comunicata agli iscritti tempestivamente.

\sphinxlineitem{\sphinxstylestrong{21.3}}
\sphinxAtStartPar
Fermo restando quanto disposto dalla normativa vigente, l’Ingegnere che rediga o abbia redatto un Piano di Governo del Territorio, un piano di fabbricazione, o altri strumenti urbanistici d’iniziativa pubblica nonché il programma pluriennale d’attuazione deve astenersi, dal momento dell’incarico fino all’approvazione, dall’accettare da committenti privati incarichi professionali inerenti l’area oggetto dello strumento urbanistico. Il periodo di tempo di incompatibilità deve intendersi quello limitato sino alla prima adozione dello strumento da parte dell’amministrazione committente. Tale norma è estesa anche a quei professionisti che con il redattore del piano abbiano rapporti di collaborazione professionale continuativa in atto.

\sphinxlineitem{\sphinxstylestrong{21.4}}
\sphinxAtStartPar
L’Ingegnere non può accettare la nomina ad arbitro o ausiliario del giudice e comunque non può assumere in qualsivoglia veste la figura di soggetto giudicante se una delle parti del procedimento sia assistita, o sia stata assistita negli ultimi due anni, da altro professionista di lui socio o con lui associato, ovvero che eserciti negli stessi locali.

\sphinxlineitem{\sphinxstylestrong{21.5}}
\sphinxAtStartPar
L’Ingegnere che abbia partecipato alla programmazione e definizione di atti e/o fasi delle procedure di evidenza pubblica aventi ad oggetto servizi tecnici è tenuto ad astenersi dal concorrere alle medesime.

\sphinxlineitem{\sphinxstylestrong{21.6}}
\sphinxAtStartPar
L’Ingegnere si deve astenere dall’assumere incarichi nei seguenti casi:
\begin{enumerate}
\sphinxsetlistlabels{\alph}{enumi}{enumii}{}{.}%
\item {} 
\sphinxAtStartPar
posizione di giudice in un concorso a cui partecipa come concorrente (o viceversa) un altro professionista che con il primo abbia rapporti di parentela o di collaborazione professionale continuativa, o tali comunque da poter compromettere l’obiettività del giudizio;

\item {} 
\sphinxAtStartPar
esercizio della professione in contrasto con norme specifiche che lo vietino e senza autorizzazione delle competenti autorità (nel caso di ingegneri dipendenti, amministratori, ecc.);

\item {} 
\sphinxAtStartPar
collaborazione sotto qualsiasi forma alla progettazione, costruzione, installazione, modifiche, riparazione e manutenzione di impianti, macchine, apparecchi, attrezzature, costruzioni e strutture per i quali riceva l’incarico di omologazione o collaudo.

\end{enumerate}

\end{description}


\subsection{Articolo 22 \sphinxhyphen{} Sanzioni}
\label{\detokenize{capitoli/codice/codice_deontologico:articolo-22-sanzioni}}\begin{description}
\sphinxlineitem{\sphinxstylestrong{22.1}}
\sphinxAtStartPar
La violazione delle norme di comportamento di cui ai precedenti articoli del presente Codice Disciplinare è sanzionata, a giudizio del Consiglio di disciplina territoriale.

\end{description}


\bigskip\hrule\bigskip



\section{\sphinxstyleemphasis{Capo VII \sphinxhyphen{} Disposizioni Finali}}
\label{\detokenize{capitoli/codice/codice_deontologico:capo-vii-disposizioni-finali}}

\subsection{Articolo 23 \sphinxhyphen{} Disposizioni finali}
\label{\detokenize{capitoli/codice/codice_deontologico:articolo-23-disposizioni-finali}}\begin{description}
\sphinxlineitem{\sphinxstylestrong{23.1}}
\sphinxAtStartPar
Il presente Codice:
\begin{enumerate}
\sphinxsetlistlabels{\alph}{enumi}{enumii}{}{.}%
\item {} 
\sphinxAtStartPar
è depositato presso il \sphinxstyleemphasis{Ministero della Giustizia}, il \sphinxstyleemphasis{Consiglio Nazionale degli Ingegneri}, gli \sphinxstyleemphasis{Ordini Provinciali}, gli \sphinxstyleemphasis{Uffici Giudiziari e Amministrativi} interessati;

\item {} 
\sphinxAtStartPar
è pubblicato sul sito ufficiale del Consiglio Nazionale e, nella versione recepita e approvata da ogni singolo Consiglio dell’Ordine, sul sito Internet di ciascun Ordine territoriale degli Ingegneri.

\end{enumerate}

\end{description}


\bigskip\hrule\bigskip


\sphinxstepscope


\chapter{Sanzioni e provvedimenti}
\label{\detokenize{capitoli/sanzioni/sanzioni:sanzioni-e-provvedimenti}}\label{\detokenize{capitoli/sanzioni/sanzioni:sanzioni}}\label{\detokenize{capitoli/sanzioni/sanzioni::doc}}
\sphinxAtStartPar
In questa pagina verranno inserite la lista delle sanzioni e dei provvedimenti disciplinari in cui si può incorrere se non viene rispettato il {\hyperref[\detokenize{capitoli/codice/codice_deontologico:codice-deontologico}]{\sphinxcrossref{\DUrole{std,std-ref}{Codice Deontologico}}}}.

\sphinxAtStartPar
\sphinxtitleref{Pagina in allestimento}

\sphinxstepscope


\chapter{Mappe Concettuali}
\label{\detokenize{capitoli/mappe/mappe:mappe-concettuali}}\label{\detokenize{capitoli/mappe/mappe:id1}}\label{\detokenize{capitoli/mappe/mappe::doc}}
\sphinxAtStartPar
In questa pagina verranno inserite le mappe concettuali utili allo studio in vista dell’esame di abilitazione alla professione di \sphinxstylestrong{Ingegnere}.

\sphinxAtStartPar
\sphinxtitleref{Pagina in allestimento}

\sphinxAtStartPar
Questa documentazione vuole essere di aiuto a chiunque voglia intraprendere il percorso da \sphinxstylestrong{Ingegnere} e debba studiare per l’Esame di Stato.

\sphinxAtStartPar
Al momento, la documentazione comprende il {\hyperref[\detokenize{capitoli/codice/codice_deontologico:codice-deontologico}]{\sphinxcrossref{\DUrole{std,std-ref}{Codice Deontologico}}}} in modo che possa essere consultato immediatamente da smartphone. Con il tempo verranno allestite le pagine delle {\hyperref[\detokenize{capitoli/sanzioni/sanzioni:sanzioni}]{\sphinxcrossref{\DUrole{std,std-ref}{Sanzioni e provvedimenti}}}} e delle {\hyperref[\detokenize{capitoli/mappe/mappe:mappe-concettuali}]{\sphinxcrossref{\DUrole{std,std-ref}{Mappe Concettuali}}}}, inoltre verranno aggiunti dei compendi ai vari articoli del Codice Deontologico, rendendo semplice la ripetizione e lo studio.

\sphinxAtStartPar
Se questo contenuto ti è risultato utile, puoi offrirmi un caffè \sphinxhref{https://www.paypal.com/donate/?hosted\_button\_id=EKASBCXHMMARG}{\sphinxincludegraphics[width=16\sphinxpxdimen]{{paypal}.png}}


\chapter{Indice}
\label{\detokenize{index:indice}}\begin{itemize}
\item {} 
\sphinxAtStartPar
{\hyperref[\detokenize{capitoli/codice/codice_deontologico:codice-deontologico}]{\sphinxcrossref{\DUrole{std,std-ref}{Codice Deontologico}}}}

\item {} 
\sphinxAtStartPar
{\hyperref[\detokenize{capitoli/sanzioni/sanzioni:sanzioni}]{\sphinxcrossref{\DUrole{std,std-ref}{Sanzioni e provvedimenti}}}}

\item {} 
\sphinxAtStartPar
{\hyperref[\detokenize{capitoli/mappe/mappe:mappe-concettuali}]{\sphinxcrossref{\DUrole{std,std-ref}{Mappe Concettuali}}}}

\end{itemize}



\renewcommand{\indexname}{Indice}
\printindex
\end{document}